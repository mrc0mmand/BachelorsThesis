%==============================================================================
% tento soubor pouzijte jako zaklad
% this file should be used as a base for the thesis
% (c) 2008 Michal Bidlo
% E-mail: bidlom AT fit vutbr cz
% Šablonu upravil / template edited by: Ing. Jaroslav Dytrych, dytrych@fit.vutbr.cz
%==============================================================================
% kodovaní: UTF-8 (zmena prikazem iconv, recode nebo cstocs)
% encoding: UTF-8 (you can change it by command iconv, recode or cstocs)
%------------------------------------------------------------------------------
% zpracování / processing: make, make pdf, make clean
%==============================================================================
% Soubory, které je nutné upravit: / Files which have to be edited:
%   xsumsa01_tls_interoperability-20-literatura-bibliography.bib - literatura / bibliography
%   xsumsa01_tls_interoperability-01-kapitoly-chapters.tex - obsah práce / the thesis content
%   xsumsa01_tls_interoperability-30-prilohy-appendices.tex - přílohy / appendices
%==============================================================================
\documentclass[english]{fitthesis} % bez zadání - pro začátek práce, aby nebyl problém s překladem
%\documentclass[english]{fitthesis} % without assignment - for the work start to avoid compilation problem
%\documentclass[zadani]{fitthesis} % odevzdani do wisu - odkazy jsou barevné
%\documentclass[english,zadani]{fitthesis} % for submission to the IS FIT - links are color
%\documentclass[zadani,print]{fitthesis} % pro tisk - odkazy jsou černé
%\documentclass[english,zadani,print]{fitthesis} % for the print - links are black
% * Je-li prace psana v anglickem jazyce, je zapotrebi u tridy pouzit 
%   parametr english nasledovne:
%   If thesis is written in english, it is necessary to use 
%   parameter english as follows:
%      \documentclass[english]{fitthesis}
% * Je-li prace psana ve slovenskem jazyce, je zapotrebi u tridy pouzit 
%   parametr slovak nasledovne:
%      \documentclass[slovak]{fitthesis}

% Základní balíčky jsou dole v souboru šablony fitthesis.cls
% Basic packages are at the bottom of template file fitthesis.cls
%zde muzeme vlozit vlastni balicky / you can place own packages here
\usepackage{ltablex}
\usepackage{tikz}
\usetikzlibrary{arrows,decorations.markings,positioning,shapes.symbols,shapes.callouts,shapes.geometric,patterns,calc} 
\usetikzlibrary{calc}

%---rm---------------
\renewcommand{\rmdefault}{lmr}%zavede Latin Modern Roman jako rm / set Latin Modern Roman as rm
%---sf---------------
\renewcommand{\sfdefault}{qhv}%zavede TeX Gyre Heros jako sf
%---tt------------
\renewcommand{\ttdefault}{lmtt}% zavede Latin Modern tt jako tt

% vypne funkci šablony, která automaticky nahrazuje uvozovky,
% aby nebyly prováděny nevhodné náhrady v popisech API apod.
% disables function of the template which replaces quotation marks
% to avoid unnecessary replacements in the API descriptions etc.
\csdoublequotesoff

% =======================================================================
% balíček "hyperref" vytváří klikací odkazy v pdf, pokud tedy použijeme pdflatex
% problém je, že balíček hyperref musí být uveden jako poslední, takže nemůže
% být v šabloně
% "hyperref" package create clickable links in pdf if you are using pdflatex.
% Problem is that this package have to be introduced as the last one so it 
% can not be placed in the template file.
\ifWis
\ifx\pdfoutput\undefined % nejedeme pod pdflatexem / we are not using pdflatex
\else
  \usepackage{color}
  \usepackage[unicode,colorlinks,hyperindex,plainpages=false,pdftex]{hyperref}
  \definecolor{links}{rgb}{0.4,0.5,0}
  \definecolor{anchors}{rgb}{1,0,0}
  \def\AnchorColor{anchors}
  \def\LinkColor{links}
  \def\pdfBorderAttrs{/Border [0 0 0] }  % bez okrajů kolem odkazů / without margins around links
  \pdfcompresslevel=9
\fi
\else % pro tisk budou odkazy, na které se dá klikat, černé / for the print clickable links will be black
\ifx\pdfoutput\undefined % nejedeme pod pdflatexem / we are not using pdflatex
\else
  \usepackage{color}
  \usepackage[unicode,colorlinks,hyperindex,plainpages=false,pdftex,urlcolor=black,linkcolor=black,citecolor=black]{hyperref}
  \definecolor{links}{rgb}{0,0,0}
  \definecolor{anchors}{rgb}{0,0,0}
  \def\AnchorColor{anchors}
  \def\LinkColor{links}
  \def\pdfBorderAttrs{/Border [0 0 0] } % bez okrajů kolem odkazů / without margins around links
  \pdfcompresslevel=9
\fi
\fi
% Řešení problému, kdy klikací odkazy na obrázky vedou za obrázek
% This solves the problems with links which leads after the picture
\usepackage[all]{hypcap}

% Informace o práci/projektu / Information about the thesis
%---------------------------------------------------------------------------
\projectinfo{
  %Prace / Thesis
  project=BP,            %typ prace BP/SP/DP/DR  / thesis type (SP = term project)
  year=2017,             %rok odevzdání / year of submission
  date=\today,           %datum odevzdani / submission date
  %Nazev prace / thesis title
  title.cs={Průběžné testování interoperability knihoven TLS/SSL},  %nazev prace v cestine ci slovenstine (dle zadani) / thesis title in czech language (according to assignment)
  title.en={Continuous Integration System for Interoperability of TLS/SSL Libraries}, %nazev prace v anglictine / thesis title in english
  %Autor / Author
  author={František Šumšal},   %cele jmeno a prijmeni autora / full name and surname of the author
  author.name={František},   %jmeno autora (pro citaci) / author name (for reference) 
  author.surname={Šumšal},   %prijmeni autora (pro citaci) / author surname (for reference) 
  %author.title.p=Bc., %titul pred jmenem (nepovinne) / title before the name (optional)
  %author.title.a=PhD, %titul za jmenem (nepovinne) / title after the name (optional)
  %Ustav / Department
  department=UITS, % doplnte prislusnou zkratku dle ustavu na zadani: UPSY/UIFS/UITS/UPGM
  %                  fill in appropriate abbreviation of the department according to assignment: UPSY/UIFS/UITS/UPGM
  %Skolitel / supervisor
  supervisor=Aleš Smrčka, %cele jmeno a prijmeni skolitele / full name and surname of the supervisor
  supervisor.name={Aleš},   %jmeno skolitele (pro citaci) / supervisor name (for reference) 
  supervisor.surname={Smrčka},   %prijmeni skolitele (pro citaci) / supervisor surname (for reference) 
  supervisor.title.p=Ing.,   %titul pred jmenem (nepovinne) / title before the name (optional)
  supervisor.title.a={Ph.D.},    %titul za jmenem (nepovinne) / title after the name (optional)
  %Klicova slova, abstrakty, prohlaseni a podekovani je mozne definovat 
  %bud pomoci nasledujicich parametru nebo pomoci vyhrazenych maker (viz dale)
  %Keywords, abstracts, declaration and acknowledgement can be defined by following 
  %parameters or using dedicated macros (see below)
  %===========================================================================
  %Klicova slova / keywords
  %keywords.cs={Klíčová slova v českém jazyce.}, %klicova slova v ceskem ci slovenskem jazyce
  %                                              keywords in czech or slovak language
  %keywords.en={Klíčová slova v anglickém jazyce.}, %klicova slova v anglickem jazyce / keywords in english
  %Abstract
  %abstract.cs={Výtah (abstrakt) práce v českém jazyce.}, % abstrakt v ceskem ci slovenskem jazyce
  %                                                         abstract in czech or slovak language
  %abstract.en={Výtah (abstrakt) práce v anglickém jazyce.}, % abstrakt v anglickem jazyce / abstract in english
  %Prohlaseni / Declaration
  %declaration={Prohlašuji, že jsem tuto bakalářskou práci vypracoval samostatně pod vedením pana ...},
  %Podekovani (nepovinne) / Acknowledgement (optional)
  %acknowledgment={Zde je možné uvést poděkování vedoucímu práce a těm, kteří poskytli odbornou pomoc.} % nepovinne
  %acknowledgment={Here it is possible to express thanks to the supervisor and to the people which provided professional help.} % optional
}

%Abstrakt (cesky, slovensky ci anglicky) / Abstract (in czech, slovak or english)
\abstract[cs]{Cílem této práce je implementace systému pro testování
Secure Socket Layer (SSL) / Transport Layer Security (TLS) knihoven
na podporovaných systémech a jeho
využití na rozšířené sadě testů pro verifikaci jejich interoperability.
Tento systém umožňuje jak průběžné testování, tak testování na
vyžádání pro specifickou verzi knihovny. Hlavním přínosem této práce je
zajištění inteoperability nejznámějších SSL/TLS knihoven již ve fázi vývoje
a detekce chyb v co nejkratším čase. Výsledky této práce ukazují nalezené
problémy na skutečných případech využití těchto knihoven a jejich dopad
na systém, kde jsou použity.}

\abstract[en]{The goal of this thesis is to implement a continuous integration
system, which allows periodic and on-demand testing of provided
Secure Socket Layer (SSL) / Transport Layer Security (TLS) libraries
on supported systems, and to show the functionality and potential of such system by extending
the existing interoperability test suite. The main benefit of this thesis
is ensuring interoperability between the most popular SSL/TLS libraries
during their development, and to discover potential issues in the shortest
possible time. Presented results show found issues discovered by combining and
using the implementation parts of this thesis on real world scenarios.}

%Klicova slova (cesky, slovensky ci anglicky) / Keywords (in czech, slovak or english)
\keywords[cs]{testování, interoperabilita, TLS, SSL, integrace, CI}
\keywords[en]{testing, interoperability, TLS, SSL, integration, CI}

%Prohlaseni (u anglicky psane prace anglicky, u slovensky psane prace slovensky)
%Declaration (for thesis in english should be in english)
%\declaration{Prohlašuji, že jsem tuto bakalářskou práci vypracoval samostatně pod vedením pana X...
%Další informace mi poskytli...
%Uvedl jsem všechny literární prameny a publikace, ze kterých jsem čerpal.}

\declaration{Hereby I declare that this bachelor's thesis was prepared as an
original author’s work under the supervision of Ing. Aleš Smrčka, Ph.D.
The supplementary information was provided by Stanislav Židek and Hubert Kario.
All the relevant information sources, which were used during preparation of this
thesis, are properly cited and included in the list of references.}

%Podekovani (nepovinne, nejlepe v jazyce prace) / Acknowledgement (optional, ideally in the language of the thesis)
%#\acknowledgment{V této sekci je možno uvést poděkování vedoucímu práce a těm, kteří poskytli odbornou pomoc
%(externí zadavatel, konzultant, apod.).}
\acknowledgment{I would like to thank Ing. Aleš Smrčka, Ph.D. for his guidance
in formal parts, structure and direction of this thesis, Stanislav Židek and Hubert Kario
for their patience and incredible technical support during the entire course
of this thesis, and Jakub Mach and Kamil Jeřábek for their moral and encouraging
support when it was needed the most.}

% řeší první/poslední řádek odstavce na předchozí/následující stránce
% solves first/last row of the paragraph on the previous/next page
\clubpenalty=10000
\widowpenalty=10000

\begin{document}
  % Vysazeni titulnich stran / Typesetting of the title pages
  % ----------------------------------------------
  \maketitle
  % Obsah
  % ----------------------------------------------
  \tableofcontents
  
  % Seznam obrazku a tabulek (pokud prace obsahuje velke mnozstvi obrazku, tak se to hodi)
  % List of figures and list of tables (if the thesis contains a lot of pictures, it is good)
\ifczech
  \renewcommand\listfigurename{Seznam obrázků}
\fi
\ifslovak
  \renewcommand\listfigurename{Zoznam obrázkov}
\fi

  % \listoffigures
\ifczech
  \renewcommand\listtablename{Seznam tabulek}
\fi
\ifslovak
  \renewcommand\listtablename{Zoznam tabuliek}
\fi

  % \listoftables 

  % vynechani stranky v oboustrannem rezimu
  % Skip the page in the two-sided mode
  \iftwoside
    \cleardoublepage
  \fi

  % Text prace / Thesis text
  % ----------------------------------------------
  %=========================================================================
% (c) Michal Bidlo, Bohuslav Křena, 2008

\chapter{Introduction}
\chapter{SSL/TLS}
    \textit{Secure Socket Layer}~(SSL) and its successor
    \textit{Transport Layer Security}~(TLS) are cryptographic protocols
    designed to provide communications security over a computer network. As
    the latest version of the SSL protocol (version~3~\cite{rfc6101}) was
    deprecated in June~2015~\cite{rfc7568} it will not be discussed
    further in this thesis.
\section{Overview}
    The primary goal of the TLS protocol is to provide privacy and data
    integrity between two communicating applications. The structure of
    the protocol comprises of two layers: the TLS Record Protocol and
    the TLS Handshake Protocol.

    The TLS Record Protocol lies at the lowest level, above some reliable
    transport protocol (e.g., TCP~\footnote{Transmission Control Protocol}).
    This protocol provides security which has following properties:
    \begin{itemize}
        \item The connection is private. Symmetric cryptography is used
        for data encryption (e.g., AES~\footnote{Advanced Encryption Standard},
        Camellia, 3DES~\footnote{Triple DES (Data Encryption Standard)}, etc.).
        The keys for the symmetric encryption are generated uniquely
        for each connection
        and are based on a secret negotiated by another protocol
        (such as the TLS Handshake Protocol). The TLS Record Protocol can also
        be used without encryption.
        \item The connection is reliable. Message transport includes
        a message integrity check using a keyed
        MAC~\footnote{Message Authentication Code}. Secure hash functions
        (e.g., SHA-1~\footnote{Secure Hash Algorithm}, SHA-256, etc.) are
        used for MAC computations. The MAC is used to prevent undetected
        data loss or data modification during transmission.
    \end{itemize}

    The TLS Record Protocol is used for encapsulation of various higher-level
    protocols. One such protocol is the TLS Handshake Protocol. This protocol
    allows client authentication and negotiation of necessary properties
    of the TLS connection, like encryption algorithm or cryptographic keys,
    before the data transmission. The TLS Handshake Protocol provides
    connection security which has following properties:
    \begin{itemize}
        \item The peer's identity can be authenticated using asymmetric, or
        public key, cryptography (e.g.,
        RSA~\footnote{Rivest-Shamir-Adleman cryptosystem},
        ECDSA~\footnote{Elliptic Curve Digital Signature Algorithm}, etc.).
        This authentication is not mandatory, but it is generally required
        for at least one of the peers.
        \item The negotiation of a shared secret is secure. Obtaining of
        the shared secred is infeasible for any eavesdropper or attacker,
        who can place himself in the middle of the authenticated connection.
        \item The negotiation is reliable. The negotiation communication
        cannot be modified without being detected by the parties to the
        communication.~\cite{rfc5246}
    \end{itemize}

    In addition to the properties above, a carefully configured TLS connection
    can provide another important privacy-related property: forward secrecy.
    This property ensures, that any future disclosure or leakage of encryption
    keys cannot be used to decrypt any TLS communnications recorded or
    eavesdropped in the past.

    TLS supports various combinations of algorithms for key exchange, data
    encryption and message integrity authentication, which is an important fact
    for this thesis and can cause severe issues when these combinations
    are configured or implemented improperly. Along with these combinations,
    TLS supports many extensions further extending its capabilities and possibilites,
    which will be described further in this thesis.

\section{TLS handshake}
- three subprotocols \\
    - security parameters negotiation \\
    - authentication \\
    - error reporting
\subsection{The Handshake Protocol}
- session negotiation \\
    - session identifier \\
    - peer certificate \\
    - compression method \\
    - cipher spec \\
    - master secret \\
    - is resumable \\
\\
- handshake: \\
    - exchange hello messages (algorithms, random values, session resumption) \\
    - exchange crypto parameters for premaster secret \\
    - exchange certs and crypto info for authentication \\
    - generate a master secret from the premaster secret and random values \\
    - provide security params to the TLS Record Layer \\
    - allow parameter verification between peers \\
\\
- types of handshake \\
    - basic \\
    - client auth \\
    - resumed \\
        - sessionID \\
        - tickets
\subsection{Change Cipher Spec Protocol}
- signal transitions in ciphering strategies \\
- ChangeCipherSpec message \\
- when recieved, read pending state is copied into the read current state \\
- after sending, sender must copy write pening state into the write active state
\subsection{Alert protocol}
- alert message has a severiry indicator (warning/fatal) and a content \\
- fatal => immediate connection termination
\subsubsection{Closure alerts}
- client must tell others that the connection is ending to prevent truncation
attacks
\subsubsection{Error alerts}
- simple error handling => when error is detected, an error message is sent
to the other party
- list of defined error alerts...

 % viz. obsah.tex / see obsah.tex

  % Pouzita literatura / Bibliography
  % ----------------------------------------------
\ifslovak
  \makeatletter
  \def\@openbib@code{\addcontentsline{toc}{chapter}{Literatúra}}
  \makeatother
  \bibliographystyle{bib-styles/czechiso}
\else
  \ifczech
    \makeatletter
    \def\@openbib@code{\addcontentsline{toc}{chapter}{Literatura}}
    \makeatother
    \bibliographystyle{bib-styles/czechiso}
  \else 
    \makeatletter
    \def\@openbib@code{\addcontentsline{toc}{chapter}{Bibliography}}
    \makeatother
    \bibliographystyle{bib-styles/englishiso}
  %  \bibliographystyle{alpha}
  \fi
\fi
  \begin{flushleft}
  \bibliography{xsumsa01_tls_interoperability-20-literatura-bibliography}
  \end{flushleft}

  % vynechani stranky v oboustrannem rezimu
  % Skip the page in the two-sided mode
  \iftwoside
    \cleardoublepage
  \fi

  % Prilohy / Appendices
  % ---------------------------------------------
  \appendix
\ifczech
  \renewcommand{\appendixpagename}{Přílohy}
  \renewcommand{\appendixtocname}{Přílohy}
  \renewcommand{\appendixname}{Příloha}
\fi
\ifslovak
  \renewcommand{\appendixpagename}{Prílohy}
  \renewcommand{\appendixtocname}{Prílohy}
  \renewcommand{\appendixname}{Príloha}
\fi
  \appendixpage

% vynechani stranky v oboustrannem rezimu
% Skip the page in the two-sided mode
\iftwoside
  \cleardoublepage
\fi
  
\ifslovak
%  \section*{Zoznam príloh}
%  \addcontentsline{toc}{section}{Zoznam príloh}
\else
  \ifczech
%    \section*{Seznam příloh}
%    \addcontentsline{toc}{section}{Seznam příloh}
  \else
%    \section*{List of Appendices}
%    \addcontentsline{toc}{section}{List of Appendices}
  \fi
\fi
  \startcontents[chapters]
  % seznam příloh / list of appendices
  % \printcontents[chapters]{l}{0}{\setcounter{tocdepth}{2}}
  
  % vynechani stranky v oboustrannem rezimu
  \iftwoside
    \cleardoublepage
  \fi
  % RFC5246 - TLS alert messages
\chapter{TLS Alerts}
    \noindent\begin{tabularx}{\linewidth}{@{}l l X}
    \caption{TLS Alerts}\label{tab:tls-alerts} \\
    \toprule
    \textbf{Alert} & \textbf{ID} & \textbf{Description} \\
    \midrule
    \endhead
    \texttt{close\_notify}                & 0   & The sender notifies the recipient that it will not send any more messages
                                                  on this connection. \\
    \texttt{unexpected\_message}          & 10  & An inappropriate message was received. This alert is always fatal. \\
    \texttt{bad\_record\_mac}             & 20  & The sender received a record with an incorrect MAC. This alert is always fatal. \\
    \texttt{decryption\_failed\_RESERVED} & 21  & Used in some earlier versions of TLS, must not be sent by compliant implementations. \\
    \texttt{record\_overflow}             & 22  & A \texttt{TLSCiphertext} record was received that had a length more than $2^{14}+2048$ bytes or
                                                  a record decrypted to a \texttt{TLSCompressed} record with more than $2^{14}+1024$ bytes. This alert
                                                  is always fatal. \\
    \texttt{decompression\_failure}       & 30  & The decompression function received improper input. This alert is always fatal. \\
    \texttt{handshake\_failure}           & 40  & The sender was unable to negotiate an acceptable set of security parameters given
                                                  the options available. This alert is always fatal. \\
    \texttt{no\_certificate\_RESERVED}    & 41  & This alert was used in SSLv3 but it no longer used in any TLS version. \\
    \texttt{bad\_certificate}             & 42  & The sender notifies the recipient that the provided certificate is corrupt. \\
    \texttt{unsupported\_certificate}     & 43  & The sender notifies the recipient that the provided certificate is of an unsupported type. \\
    \texttt{certificate\_revoked}         & 44  & The sender notifies the recipient that the provided certificate was revoked by the issuing authority. \\
    \texttt{certificate\_expired}         & 45  & The sender notifies the recipient that the provided certificate has expired or is no longer valid. \\
    \texttt{certificate\_unknown}         & 46  & The sender notifies the recipient that some unspecified issue occured during the certificate processing,
                                                  rendering it unacceptable. \\
    \texttt{illegal\_parameter}           & 47  & A field in the handshake was out of range or inconsistent with other fields. This alert is always fatal. \\
    \texttt{unknown\_ca}                  & 48  & The received certificate could not be validated, because the CA certificate could not be located or could
                                                  not be matched with a known, trusted CA. This alert is always fatal. \\
    \texttt{access\_denied}               & 49  & A valid certificate was received, but when access control was applied, the sender decided not to proceed
                                                  with negotiation. This alert is always fatal. \\
    \texttt{decode\_error}                & 50  & The received message could not be decoded because some field was out of the specified range or length of
                                                  the message was incorrect. This alert is always fatal. \\
    \texttt{decrypt\_error}               & 51  & A handshake cryptographic operation failed. This alert is always fatal. \\
    \texttt{export\_restriction\_RESERVED}& 60  & Used in some earlier versions of TLS, must not be sent by compliant implementations. \\
    \texttt{protocol\_version}            & 70  & The protocol version the client has attempted to negotiate is recognized but not supported.
                                                  This alert is always fatal. \\
    \texttt{insufficient\_security}       & 71  & The server requires more secure ciphers than those supported by the client. This alert is always fatal. \\
    \texttt{internal\_error}              & 80  & An internal error occured, unrelated to the peer or corectness of the protocol. This alert is always fatal. \\
    \texttt{user\_canceled}               & 90  & This handshake is being canceled for some reason unrelated to a protocol failure. This alert should be followed
                                                  by a \texttt{close\_notify}. \\
    \texttt{no\_renegotiation}            & 100 & The peer should respond with this alert when renegotiation is not appropriate regarding the current connection
                                                  state. This alert is always a warning. \\
    \texttt{unsupported\_extension}       & 110 & Sent by the client when the received \texttt{ServerHello} message contains an extension not sent by the client
                                                  in its \texttt{ClientHello} message. This alert is always fatal.
    \end{tabularx}


\chapter{Test Plan}
\section{Test Plan Identifier}
    TLS/SSL Interoperability Test Plan v0.1

\section{References}
    \begin{itemize}
        \item IEEE 829-2008 Standard for Software Test
        Documentation~\footnote{http://standards.ieee.org/findstds/standard/829-2008.html}
        \item Common Criteria \@ access.redhat.com~\footnote{https://access.redhat.com/blogs/766093/posts/1976523}
    \end{itemize}

\section{Introduction}
    The main goal of this test plan is to ensure interoperability of supported
    SSL/TLS libraries on CentOS/RHEL and Fedora systems. The testing itself involves
    verification of ability to comunicate between two libraries using various
    combination of cipher suites, connection settings and extensions.

\section{Test Items}
\subsection{Components}
    \begin{itemize}
        \item OpenSSL
        \item NSS
        \item GnuTLS
    \end{itemize}

\subsection{Environments: Releases and Architectures}
    Due to limitations of the current CI all tests are run on x86\_64 architeture
    only. Nevertheless, they should work on all architectures supported by
    the underlying operating system.

    \begin{itemize}
        \item CentOS 6 and 7 (latest releases)
        \item Fedora (latest release)
    \end{itemize}

\section{Software Risk Issues}
    \begin{itemize}
        \item Package rebases can cause unexpected behavior and/or regressions
              -- thorough test results analysis is necessary
    \end{itemize}

\section{Features to be Tested}
    All features are tested using TLSv1.1 and TLSv1.2 protocols with all supported
    cipher suites by the involved parties.

    \begin{itemize}
        \item Basic interoperability
        \item Inteoperability with client certificates
        \item Session renegotiation
        \item Session renegotiation with client certificates
        \item Session resumption using Session ID
        \item Session resumption using Session ID with client certificates
        \item Session resumption using TLS SessionTicket Extension
        \item Session resumption using TLS SessionTicket Extension with client certificates
        \item SignatureAlgorithms TLS Extension
    \end{itemize}

\section{Features not to be Tested}
    \begin{itemize}
        \item Sanity of the available options
        \item Regressions
        \item Security of the implementation
    \end{itemize}

\section{Approach}
    All testing is done by Bash scripts using
    Beakerlib~\footnote{https://github.com/beakerlib/beakerlib} testing framework.
    This framework manages log collection and results reporting and automatizes
    the entire testing process.

    Each library has a set of utilites, which are used for the interoperability
    testing itself:

    \begin{itemize}
        \item \textbf{OpenSSL} -- \texttt{openssl} utility (package \textit{openssl})
        \item \textbf{NSS} -- utilities \texttt{selfserv}, \texttt{tstclnt},
              \texttt{strsclnt}, etc. (package \textit{nss-tools})
        \item \textbf{GnuTLS} -- utilities \texttt{gnutls-cli} and
              \texttt{gnutls-serv} (package \textit{gnutls-utils})
    \end{itemize}

    All libraries are tested in pairs in a client-server fashion, where each
    phase tests a specific combination of parameters (specific cipher suite,
    protocol, extension, etc.).

    Each failure is investigated and if it is a library issue, it is reported
    to the upstream and/or to a respective downstream bug tracker.

\section{Item Pass/Fail Criteria}
    A handshake is completed successfully in all cases with all requested
    settings set, i.e.:
    \begin{itemize}
        \item Expected cipher suite is used
        \item Expected protocol is used
        \item A specific extension requested in Client/Server Hello is used
        \item Session is correctly resumed when session resumption is requested
        \item Session renegotiation is successful
    \end{itemize}

\section{Test Cases}
    \noindent\begin{tabularx}{\linewidth}{@{}X c c c }
    \caption{Test case matrix}\label{tab:test-plan-matrix} \\
    \toprule
    \textbf{Test case} & \textbf{CentOS 6} & \textbf{CentOS 7} & \textbf{Fedora} \\
    \midrule
    \endhead
    gnutls/renegotiation-with-NSS                 &   & x & x \\
    gnutls/renegotiation-with-OpenSSL             &   & x & x \\
    gnutls/resumption-with-NSS                    &   & x & x \\
    gnutls/resumption-with-OpenSSL                &   & x & x \\
    gnutls/signature\_algorithms-with-OpenSSL     &   & x & x \\
    gnutls/softhsm-integration                    &   & x & x \\
    gnutls/TLSv1-2-with-NSS                       & x & x & x \\
    gnutls/TLSv1-2-with-OpenSSL                   & x & x & x \\
    \hline
    nss/CC-nss-with-gnutls                        &   & x & x \\
    nss/CC-nss-with-openssl                       &   & x & x \\
    nss/Interoperability-with-OpenSSL             & x & x & x \\
    nss/renego-and-resumption-NSS-with-OpenSSL    & x & x & x \\
    \hline
    openssl/CC-openssl-with-gnutls                &   & x & x \\
    \end{tabularx}

    \noindent Brief description of all test cases can be found
    in Appendix~\ref{ref:test-case-description}.

\section{Suspension Criteria and Resumption Requirements}
    Testing will be suspended if any of the following criteria are met:
    \begin{itemize}
        \item Underlying operating system is not installable
        \item Existing issues may prevent execution of the test suite
    \end{itemize}

    Testing will be resumed when all mentioned issues are resolved.

\section{Test Deliverables}
    The test results generated by Beakerlib will be stored in the CI and
    all failures will be analysed. The analysis itself can yield following
    results:

    \begin{itemize}
        \item Failure caused by the tested component -- a new bug will be reported
        \item Failure caused by the test -- the test case will be fixed
        \item Failure caused by an error in the infrastructure/environment --
              the test case will be run again
    \end{itemize}

\section{Remaining Test Tasks}
    Extend test coverage to other TLS extensions:
    \begin{itemize}
        \item extended\_master\_secret extension
        \item encrypt\_then\_mac extension
        \item etc.
    \end{itemize}

    Implementation of tests for these extension is currently blocked on the
    limited support of these extensions by the utilities of SSL/TLS libraries.

    Testing of some recent algorithms for TLS should be considered as well
    (e.g. \textit{ChaCha20-Poly1305}~\footnote{https://www.rfc-editor.org/rfc/rfc7905.txt}).
    This will have to wait until the SSL/TLS libraries provide support for
    these algorithms.

\section{Environmental Needs}
\subsection{Hardware}
    Testing will be performed on x86\_64 architecture as it is the only architecture
    supported by the current CI. Particular hardware configuration is not important
    for the testing itself.

\subsection{Software}
    No special configuration of the operating system is needed. All packages
    necessary for the testing will be installed by the CI system.

\section{Staffing and Training needs}
    N/A

\section{Responsibilities}
    \begin{itemize}
        \item František Šumšal
        \item Stanislav Židek
        \item Hubert Kario
    \end{itemize}

\section{Schedule}
    \todo{Update schedule => +periodic runs, +devel branches}
    Currently all tests are being executed when a new PR or commit is pushed
    to the test repository. In the nearest future, all tests should be executed
    periodically, and when a new version of a supported system is released.

    Long term plans include delivering all test cases to both downstream and
    upstream, so possible failures can be detected before the library itself
    is released.

\section{Approvals}
    \begin{itemize}
        \item Stanislav Židek
    \end{itemize}

\chapter{Test Cases Description} \label{ref:test-case-description}
    Note: the term "various cipher suites" used in following sections describes
    cipher suites using different key exchange algorithms (RSA, DHE, ECDHE),
    authentication algorithms (RSA, DSA, ECDSA), block cipher algorithms
    (3DES, AES) and message authentication algorithms (SHA). Mentioned
    algorithms may also differ in modes (AES-GCM, AES-CBC, \dots) and sizes
    (AES-128, AES-256, SHA-1, SHA-256, \dots).
\section{GnuTLS}
    \begin{description}
        \item[renegotiation-with-NSS] \hfill \\
            Test session renegotiation between GnuTLS and NSS libraries using
            various cipher suites, TLSv1.1 and TLSv1.2 protocols, and
            client certificates.
        \item[renegotiation-with-OpenSSL] \hfill \\
            Test session renegotiation between GnuTLS and OpenSSL libraries using
            various cipher suites, TLSv1.1 and TLSv1.2 protocols, and
            client certificates.
        \item[resumption-with-NSS] \hfill \\
            Test session resumption between GnuTLS and NSS libraries using
            various cipher suites, TLSv1.1 and TLSv1.2 protocols, and
            client certificates. The resumption itself is tested using both
            Session IDs and SessionTicket extension.
        \item[resumption-with-OpenSSL] \hfill \\
            Test session resumption between GnuTLS and OpenSSL libraries using
            various cipher suites, TLSv1.1 and TLSv1.2 protocols, and
            client certificates. The resumption itself is tested using both
            Session IDs and SessionTicket extension.
        \item[signature\_algorithms-with-OpenSSL] \hfill \\
            Test signature\_algorithms extension in communication between
            GnuTLS and OpenSSL libraries with and without client certificates.
            This extension is present only in TLSv1.2 and higher.
            As this extension
            has to be explicitly enabled, it is currently tested only in
            this combination of libraries, due to lack of support in testing
            utilities provided by the libraries.
        \item[TLSv1-2-with-NSS] \hfill \\
            Verify interoperability of GnuTLS with NSS using TLSv1.2 protocol
            with various cipher suites.
        \item[TLSv1-2-with-OpenSSL] \hfill \\
            Verify interoperability of GnuTLS with OpenSSL using TLSv1.2 protocol
            with various cipher suites.
    \end{description}
\section{NSS}
    \begin{description}
        \item[CC-nss-with-gnutls] \hfill  \\
            Test interoperability of cipher suites relevant for Common Criteria
            certification between NSS and GnuTLS libraries with and without
            client certificates.
        \item[CC-nss-with-openssl] \hfill  \\
            Test interoperability of cipher suites relevant for Common Criteria
            certification between NSS and OpenSSL libraries with and without
            client certificates.
        \item[Interoperability-with-OpenSSL] \hfill  \\
            Test interoperability between NSS and OpenSSL libraries using
            certificates generated by NSS (all other tests use certificates
            generated by OpenSSL). This test is currently under heavy
            refactoring as it was originally developed for RHEL 6.
        \item[renego-and-resumption-NSS-with-OpenSSL] \hfill  \\
            Test session renegotiaton and session resumption between NSS and
            OpenSSL libraries. Both methods are tested using various cipher
            suites, TLSv1.1 and TLSv1.2 protocols, and client certificates.
            Moreover, session resumption is testes using both Session IDs and
            SessionTicket extension.
    \end{description}
\section{OpenSSL}
    \begin{description}
        \item[CC-openssl-with-gnutls] \hfill \\
            Test interoperability of cipher suites relevant for Common Criteria
            certification between OpenSSL and GnuTLS libraries with and without
            client certificates.
    \end{description}
 % viz. prilohy.tex / see prilohy.tex
\end{document}
