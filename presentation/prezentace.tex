\documentclass[10pt,xcolor=pdflatex]{beamer}
\usepackage{newcent}
\usepackage[utf8]{inputenc}
\usepackage[czech]{babel}
\usepackage{hyperref}
\usepackage{fancyvrb}
\usetheme{FIT}

%%%%%%%%%%%%%%%%%%%%%%%%%%%%%%%%%%%%%%%%%%%%%%%%%%%%%%%%%%%%%%%%%%
\title[Průběžné testování interoperability knihoven TLS/SSL]{Průběžné testování interoperability knihoven TLS/SSL}

\author[]{František Šumšal}

\institute[]{Vysoké Učení Technické v Brně, Fakulta Informačních Technologií\\
Bo\v{z}et\v{e}chova 1/2. 612 66 Brno - Kr\'alovo Pole\\
xsumsa01@stud.fit.vutbr.cz}

%\date{January 1, 2016}
\date{\today}
%\date{} % bez data

%%%%%%%%%%%%%%%%%%%%%%%%%%%%%%%%%%%%%%%%%%%%%%%%%%%%%%%%%%%%%%%%%%

\begin{document}

\frame[plain]{\titlepage}

\section{Zadání}

\begin{frame}
\frametitle{Zadání}
\begin{enumerate}
    \item Seznamte se s projekty OpenSSL, NSS a GnuTLS implementující protokol SSL/TLS. Zaměřte se na testování interoperability těchto projektů.
    \item Navrhněte systém pro testování interoperability a integrační testování programů a knihoven implementující SSL/TLS protokoly. Jedním z cílů systému je poskytovat testování navrhovaných změn daných projektů ještě před jejich přijetím. Systém by měl podporovat testy napsané s podporou projektu BeakerLib.
    \item Implementujte systém pro testování interoperability a integračního testování. Importujte stávající testovací sady do nově implementovaného systému.
    \item Kvalitu dosaženého řešení demonstrujte na příkladech s uměle vytvořenými chybami.
\end{enumerate}
\end{frame}

%------------------------------------------------

\section{Implementace}

\begin{frame}
\frametitle{Testování interoperability}
\begin{itemize}
    \item Možná implementace vlastních testovacích utilit pomocí public API knihoven - zbytečné, několik verzí API, nutná údržba
    \item Každá knihovna poskytuje vlastní sadu utilit (openssl, gnutls-cli, selfserv, \dots)
    \item Využití těchto utilit v testování (velké množství přepínačů/voleb, detailní analýza komunikace, \dots)
\end{itemize}
\end{frame}

%------------------------------------------------

\begin{frame}
\frametitle{Testovací sada}
\begin{itemize}
    \item Využit testovací framework BeakerLib (Bash)
    \item Původní sada testů rozšířena (rozšíření stávajících testů + implementace nových)
    \item Testování knihoven v párech (sady šifer, protokoly, certifikáty, \dots)
\end{itemize}
\end{frame}

%------------------------------------------------

\begin{frame}
\frametitle{Průběžné testování (CI)}
\begin{itemize}
    \item Několik funkčních prototypů (Webhooks, Jenkins, \dots)
    \item Finální návrh - Travis CI (open-source, podpora GitHub repozitářů, \dots)
    \item Nekompatibilní OS (Ubuntu vs CentOS/Fedora) $\implies$ Docker
    \item Další problémy (RHEL vs CentOS, FIPS, limity Travis CI, \dots)
    \item Kompilace knihovny před testováním
\end{itemize}
\end{frame}

%------------------------------------------------

\section{Výsledky}

\begin{frame}
\frametitle{Výsledky}
\begin{itemize}
    \item Během vývoje a rozšiřování sady testů nalezeno několik chyb (knihovny, BeakerLib, \dots)
    \item Jedno CVE (CVE-2016-9574)
    \item Chyby ohlášeny a potvrzeny odpovědným stranám (upstream, downstream)
\end{itemize}
\end{frame}

%------------------------------------------------

\section{Budoucí vývoj}

\begin{frame}
\frametitle{Budoucí vývoj}
\begin{itemize}
    \item Zdrojový kód je dostupný na GitHubu - \url{https://github.com/redhat-qe-security/interoperability}
    \item Systém průběžné integrace - \url{https://travis-ci.org/redhat-qe-security/interoperability}
    \item Naplánován další vývoj a údržba projektu
    \item Spolupráce s vývojáři podporovaných knihoven
\end{itemize}
\end{frame}

%------------------------------------------------

\bluepage{Děkuji za pozornost}

%------------------------------------------------

\section{Otázky}

\begin{frame}
\frametitle{Otázky}
\begin{block}{Otázka č. 1}
Předpokládejte, že jsem autorem nové implementace TLS/SSL (nebo např. zmiňovaného BoringSSL). Jaké kroky je třeba podniknout, aby tato implementace byla podporována Vašim řešením?
\end{block}

\begin{itemize}
    \item Zajistit, že je daná implementace přeložitelná na podporovaných systémech
    \item Rozšířit sadu testů o testy využívající danou implementaci (+CC)
\end{itemize}
\end{frame}

\begin{frame}
\frametitle{Otázky}
\begin{block}{Otázka č. 2}
Je Vaše řešení multiplatformní? Bylo by možné testovat interoperabilitu knihoven např. na MacOS X?
\end{block}

\begin{itemize}
    \item Testování probíhá v Docker kontejnerech
    \item Nutné přípravy (klonování repozitářů, příprava knihoven, kontrola testů, \dots)
    \item Na MacOS teoreticky bez úprav, na jiných systémech je třeba zajistit prvotní přípravy
\end{itemize}
\end{frame}

\end{document}
