%=========================================================================
% (c) Michal Bidlo, Bohuslav Křena, 2008

\chapter{Introduction}
\chapter{SSL/TLS}
    \textit{Secure Socket Layer}~(SSL) and its successor
    \textit{Transport Layer Security}~(TLS) are cryptographic protocols
    designed to provide communications security over a computer network. As
    the latest version of the SSL protocol (version~3.0~\cite{rfc6101}) was
    deprecated in June~2015~\cite{rfc7568} it will not be discussed
    further in this thesis.
\section{Overview}\label{overview}
    The primary goal of the TLS protocol is to provide privacy and data
    integrity between two communicating applications. The structure of
    the protocol comprises two layers: the TLS Record Protocol and
    the TLS Handshake Protocol.

    The TLS Record Protocol lies at the lowest level, above some reliable
    transport protocol (e.g., TCP~\footnote{Transmission Control Protocol}).
    This protocol provides security which has following properties:
    \begin{itemize}
        \item The connection is private. Symmetric cryptography is used
        for data encryption (e.g., AES~\footnote{Advanced Encryption Standard},
        Camellia, 3DES~\footnote{Triple DES (Data Encryption Standard)}, etc.).
        The keys for the symmetric encryption are generated uniquely
        for each connection
        and are based on a secret negotiated by another protocol
        (such as the TLS Handshake Protocol). The TLS Record Protocol can also
        be used without encryption.
        \item The connection is reliable. Message transport includes
        a message integrity check using a keyed
        MAC~\footnote{Message Authentication Code}. Secure hash functions
        (e.g., SHA-1~\footnote{Secure Hash Algorithm}, SHA-256, etc.) are
        used for MAC computations. The MAC is used to prevent undetected
        data loss or data modification during transmission.
    \end{itemize}

    The TLS Record Protocol is used for encapsulation of various higher-level
    protocols. One such protocol is the TLS Handshake Protocol. This protocol
    allows client authentication and negotiation of necessary properties
    of the TLS connection, like encryption algorithm or cryptographic keys,
    before the data transmission. The TLS Handshake Protocol provides
    connection security which has following properties:
    \begin{itemize}
        \item The peer's identity can be authenticated using asymmetric, or
        public key, cryptography (e.g.,
        RSA~\footnote{Rivest-Shamir-Adleman cryptosystem},
        ECDSA~\footnote{Elliptic Curve Digital Signature Algorithm}, etc.).
        This authentication is not mandatory, but it is generally required
        for at least one of the peers.
        \item The negotiation of a shared secret is secure. Obtaining of
        the shared secred is infeasible for any eavesdropper or attacker,
        who can place himself in the middle of the authenticated connection.
        \item The negotiation is reliable. The negotiation communication
        cannot be modified without being detected by the parties to the
        communication.~\cite{rfc5246}
    \end{itemize}

    In addition to the properties above, a carefully configured TLS connection
    can provide another important privacy-related property: forward secrecy.
    This property ensures, that any future disclosure or leakage of encryption
    keys cannot be used to decrypt any TLS communnications recorded or
    eavesdropped in the past.

    TLS supports various combinations of algorithms for key exchange, data
    encryption and message integrity authentication, which is an important fact
    for this thesis and can cause severe issues when these combinations
    are configured or implemented improperly. Along with these combinations,
    TLS supports many extensions further extending its capabilities and possibilites,
    which will be described further in this thesis.

    \todo{RSA, DH(E), DSS, ECDS, ECDH(E), 3DES, RC4, AES, SHA, CHACHA20, POLYXXX}

\section{TLS Protocols}
    As mentioned above, the TLS protocol consists of four protocols --
    the TLS Record Protocol, the TLS Handshake Protocol, the TLS Changer Cipher
    Spec Protocol, the TLS Alert Protocol and the TLS Application Data Protocol.
    In this section we will overview and discuss these protocols in more detail.

\subsection{TLS Record Protocol}
    In section \ref{overview} we briefly described the main function of the
    TLS Record Protocol, which is encapsulation of protocol data from higher
    layers. This includes fragmentation, optional compression, MAC application,
    encryption and transmission of the data. On the receiving side the data is
    decrypted, verified, decompressed, reassembled and then delivered to the
    higher layers.

    Through the data processing, following TLS data structures are used --
    \textit{TLSPlaintext}, \textit{TLSCompressed} and \textit{TLSCiphertext}.
    At the end a TLS record is formed by appending an TLS record header to
    the \textit{TLSCiphertext} structure.

    \todo{Describe fragmentation, compression and encryption?}
\subsection{TLS Handshake Protocol}
- session negotiation \\
    - session identifier \\
    - peer certificate \\
    - compression method \\
    - cipher spec \\
    - master secret \\
    - is resumable \\
\\
- handshake: \\
    - exchange hello messages (algorithms, random values, session resumption) \\
    - exchange crypto parameters for premaster secret \\
    - exchange certs and crypto info for authentication \\
    - generate a master secret from the premaster secret and random values \\
    - provide security params to the TLS Record Layer \\
    - allow parameter verification between peers \\
\\
- types of handshake \\
    - basic \\
    - client auth \\
    - resumed \\
        - sessionID \\
        - tickets
\subsection{TLS Change Cipher Spec Protocol}
- signal transitions in ciphering strategies \\
- ChangeCipherSpec message \\
- when recieved, read pending state is copied into the read current state \\
- after sending, sender must copy write pening state into the write active state
\subsection{Alert protocol}
- alert message has a severiry indicator (warning/fatal) and a content \\
- fatal => immediate connection termination
\subsubsection{Closure alerts}
- client must tell others that the connection is ending to prevent truncation
attacks
\subsubsection{Error alerts}
- simple error handling => when error is detected, an error message is sent
to the other party
- list of defined error alerts...
\subsection{TLS Application Data Protocol}
- todo

\chapter{Continuous Integration}
\todo{(overview, jenkins, travis, semaphore, beaker, openstack, ...)}

\chapter {SSL/TLS Testing}
    Before the testing itself, we have to know what should be tested, how it
    should be tested and where, or to be more precise, which environments it
    should be tested in.

\section{Tested Libraries}
    To prevent each project implementing the SSL/TLS on its own and introducing
    (in many situations) dangerous issues, several libraries were created and
    can be used by any project, which needs a SSL/TLS support. The most
    popular ones are OpenSSL~\footnote{https://www.openssl.org/},
    NSS~\footnote{https://developer.mozilla.org/en-US/docs/Mozilla/Projects/NSS}
    and GnuTLS~\footnote{https://www.gnutls.org/}. Although, there are other
    SSL/TLS libraries (e.g. LibreSSL~\footnote{https://www.libressl.org/} or
    BoringSSL~\footnote{https://boringssl.googlesource.com/boringssl/}),
    this thesis aims only on these three. However, a future
    expansion to support other libraries is not impossible.

    Even though these libraries are
    separate projects, an user must be able to communicate with every client,
    which supports the particular protocol and ciphersute, no matter which
    implementation they use. Testing of this functionality - \textit{interoperability} -
    is the main goal of this thesis.

    For the testing itself we need at least two applications - a client and a server.
    One option would be writing these applications using the public
    API~\footnote{Application Programming Interface} of each library,
    which is not error prone and would require a maintenance of such applications.
    Thankfully, each of the tested libraries provides a set of utilities,
    among which we can found a simple client and server application with
    dozens of settings and options. These applications are then used in
    various scenarios to ensure, that given valid combination of settings works
    for both client and a server using different libraries.

\subsection{OpenSSL}
    OpenSSL is an open source library maintained by The OpenSSL Project,
    which provides a toolkit for TLS and SSL
    protocols, along with other general-purpose cryptographic functions.

    This library provides a single powerful utility called \textit{openssl}. This
    utility has dozens of subcommands with various SSL/TLS related functionality,
    where the most important ones are:
    \begin{description}
        \item [ciphers] information about supported ciphersuites
        \item [dsa, rsa, ec] DSA/RSA/EC key management
        \item [s\_client] a simple SSL/TLS client
        \item [s\_server] a simple SSL/TLS server
        \item [x509] X.509 certificate data management
    \end{description}

    Thanks to its functionality-rich interface, OpenSSL is used for certificate
    generation and management for all libraries in the testing process.

\subsection{NSS}
    Network Security Services~(NSS) is a set of open source libraries providing
    support for SSL and TLS protocols,
    S/MIME~\footnote{Secure/Multipurpose Internet Mail Extensions}
    and other optional features, like
    server-side TLS/SSL acceleration or client-side hardware smart cards support.

    Compared to OpenSSL, which has a single utility for everything, NSS does
    the exact opposite - each feature or tool has its own utility. For our
    testing purposes, we will need the following ones:
    \begin{description}
        \item [certutil] certificate and key management for NSS databases
        \item [listsuites] information about supported ciphersuites
        \item [selfserv] a simple SSL/TLS server
        \item [strsclnt] a simple SSL/TLS client for performance testing
        \item [tstclnt] a simple SSL/TLS client
    \end{description}

    This library differs from the other two in the way how it handles
    server and client certificates. These certificates cannot be passed
    directly as arguments to the utility, but must be imported to a NSS
    database which is then passed as an argument to the utility.

\subsection{GnuTLS}
    GnuTLS is a secure communications library which implements SSL, TLS and
    DTLS~\footnote{Datagram Transport Layer Security} protocols.

    Like the the libraries above, GnuTLS includes several utilities for library
    testing - GnuTLS client \textit{gnutls-cli} and GnuTLS server
    \textit{gnutls-serv}. Both utilities support a parameter \textbf{-l},
    which lists all necessary information about supported ciphersuites.

\section{Tested Environments}
    For the purposes of this thesis, an environment consists of an operating
    system (e.g.
    CentOS~\footnote{https://www.centos.org/},
    Fedora~\footnote{https://getfedora.org/}) and its version
    (e.g. 7, 25). Each of these environments must be tested separately, as
    it contains a different set of library versions and policies, which
    affect the SSL/TLS communication.

    This thesis covers SSL/TLS libraries on CentOS and Fedora, as the tests used
    and extended by this thesis were originaly created for
    RHEL~\footnote{Red Hat Enterprise Linux -
    https://www.redhat.com/en/technologies/linux-platforms/enterprise-linux}.


    \todo{beakerlib, testplan(!), test format, relevancy, tested systems, tested libraries,
    what's not covered and why, ...}

\chapter {Testing Results}
\todo{filed bugs/CVEs, split upstream and downstream}

\chapter {Conclusion}
\todo{summarise findings, current CI state, future work (if any), ...}
